\documentclass[compress,10pt,aspectratio=169]{beamer}
\usetheme[customnumbering]{onera}

\usepackage{amsmath,amsfonts,graphicx}
\usepackage{pifont}
\usepackage{etoolbox}
\usepackage{multicol}
\usepackage{anyfontsize}
\usepackage{multirow}
\usepackage{hyperref}
\usepackage{amsmath}
\usepackage{colortbl}
%\setlength{\columnseprule}{1pt}
%\def\columnseprulecolor{\color{blue}}
\usepackage{minted} % syntax coloring. 
\setminted{encoding=utf-8, autogobble}
\usemintedstyle{xcode}
\AtBeginEnvironment{minted}{\fontsize{8}{8}\selectfont}

%\usepackage{dsfont}
\usepackage{ifdraft}
\ifdraft{
  \usepackage{fancyvrb}
  \DefineVerbatimEnvironment{cppcode}{Verbatim}{}
}{
\newminted{cpp}{}
}
\usepackage{hyperref}
\usetikzlibrary{shadows, arrows, decorations.pathmorphing, fadings, shapes.arrows, positioning, calc, shapes, fit, matrix,math}

\definecolor{lightblue}{RGB}{0,200,255} 
\definecolor{paper}{RGB}{255,247,197}
\definecolor{ocre}{RGB}{243,102,25} % Define the orange color used for highlighting throughout the book
\definecolor{BurntOrange}{RGB}{238,154,0}
\definecolor{darkorange}{RGB}{119, 77, 0}
\definecolor{OliveGreen}{RGB}{188,238,104}
\definecolor{DarkGreen}{RGB}{0,128,0}
\definecolor{BrickRed}{RGB}{238,44,44}
\definecolor{Tan}{RGB}{210,180,140}
\definecolor{Aquamarine}{RGB}{127,255,212}
\definecolor{NavyBlue}{RGB}{0,64,128}
\definecolor{DarkYellow}{RGB}{192,192,0}
\definecolor{Yellow}{RGB}{255,255,0}

\title[Parallel programming\hspace{2em}]{Parallel performance measure and embarrassingly Parallel algorithms}
\subtitle{Performance measure and load balancing}
\author[X. JUVIGNY]{Xavier JUVIGNY, SN2A, DAAA, ONERA\\ \href{mailto:xavier.juvigny@onera.fr}{\texttt{xavier.juvigny@onera.fr}} }
\date[01/08/2023]{Course Parallel Programming\\- January 8th 2023 -}
\institute{\inst{1}ONERA,\inst{2}DAAA}

\AtBeginSection[]{
  \begin{frame}{Overview}
  \begin{multicols}{2}
  \small \tableofcontents[currentsection, hideallsubsections]
  \end{multicols}
  \end{frame} 
}

\begin{document}

\MakeTitlePage

\begin{frame}
\frametitle{Table of contents}
\begin{multicols}{2}
\tableofcontents[hideallsubsections]
\end{multicols}
\end{frame}

\section{Performance tools}

\begin{frame}{Speed-up}
    \scriptsize
    \begin{block}{Definition}
        Let's
        \begin{itemize}
        \item $t_{s}$ : Sequential execution time
        \item $t_{p}(n)$ : Execution time on $n$ computing units;
        \end{itemize}

        Speed-up is defined as :        
        \begin{equation}
         S(n) = \frac{t_{s}}{t_{p}(n)}
        \end{equation}
    \end{block}
        
    \begin{alertblock}{Remark}
        The sequential algorithm is often different as parallel algorithm. In this case,
        speed-up measure is not obvious. In particular, following questions must be 
        asked among other questions :
        \begin{itemize}
            \item Is the sequential algorithm optimal in complexity ?
            \item Is the sequential algorithm well optimized ?
            \item Is the sequential algorithm exploiting at best the cache memory ?
        \end{itemize} 
    \end{alertblock}
        
\end{frame}

\begin{frame}[fragile]{Amdahl's law}
\scriptsize
\begin{center}\textcolor{DarkGreen}{\large Give a limit for the speed-up}\end{center}

\begin{itemize}
\item Let's $t_{s}$ the time running the code in sequential
\item Let's $f$ the part (in percent) of $t_{s}$, relative to a part of the code which \textcolor{orange}{can't be parallelized} 
\end{itemize}

So
\[
    S(n) = \frac{t_{s}}{f.t_{s}+\frac{(1-f)t_{s}}{n}} = \frac{n}{1+(n-1)f} \xrightarrow[n \to \infty]{} \frac{1}{f}
\]

This law is useful to find the maximal number of computing core to use for an application.

\begin{alertblock}{Limitation of the law}
    $f$ may change with the volume of input data and bigger input data improve the speed-up.
\end{alertblock}
\end{frame}

\begin{frame}[fragile]{Gustafson's law}
\scriptsize
\begin{center}{\large \textcolor{orange}{Speed-up behaviour with constant volume input data per process}}\end{center}

\begin{itemize}
\item \textbf{Hypotheses} : 
\begin{itemize}
    \item {\scriptsize $t_{s}\geq 0$ the time to execute the sequential part of the code is indenpendant of the volume of input data;}
    \item {\scriptsize $t_{p} > 0$ the time to execute the parallel part of the code is linear relative of the volume of input data.}
    \item {\scriptsize Let's consider $t_{s}+t_{p} = 1$ (one unit of time)}.
\end{itemize}
\item Let's $t_{s}$ the time token by the execution of the sequential part of the code;
\item Let's $t_{p}$ the time token by the execution of the parallel part of the code for a fixed amount of data.
\end{itemize}

\[
    S(n) = \frac{t_{s}+n.t_{p}}{t_{s}+t_{p}} = n + \frac{(1-n)t_{s}}{t_{s}+t_{p}}
                                             = n + (1-n).t_{s}
\]
\end{frame}

\begin{frame}[fragile]{Scalability}
    \begin{block}{Definition}
        For a parallel program, the behavious of the speed-up when we raises up the number of processes or/and the amount 
        of input data. 
    \end{block}

    \begin{exampleblock}{How evaluate the scalability ?}
        \begin{itemize}
            \item Evaluate the worst speed-up : \textcolor{red}{For a global fixed amount of data}, draw the speed-up curve in function of the number of processes;
            \item Evaluate the best speed-up : \textcolor{DarkGreen}{For a fixed amount of data per process}, draw the speed-up curve in function of the number of processes;
            \item In concrete use of the program, the speed-up may be between the worst and best scenario.
        \end{itemize}
    \end{exampleblock}
\end{frame}

\begin{frame}[fragile]{Granularity}
    \scriptsize
    \begin{center}\small Ratio between computing intensity and quantity of data exchanged between processes\end{center}

    \begin{itemize}
        \item Sending and receiving data is prohibitive :
        \begin{itemize}
            \item {\scriptsize \textcolor{red}{Initial cost of a message} : each message has an initial cost : set the connection, 
                   get the same protocol, and so.\textcolor{blue}{This cost is constant.}} 
            \item {\scriptsize \textcolor{blue}{Cost of the data transfer} : At last, the cost of the data flow is linear with the number
            of data to exchange}
            \item {\scriptsize These costs are greater the cost of memory operations in RAM}
            \item \textcolor{DarkGreen}{\scriptsize Better to copy some sparse data in a buffer and send the buffer than send data with multiple send and receives}
        \end{itemize}
        \item \textcolor{darkorange}{Try to minimize the number of data exchange between processes}
        \item Greater is the ratio between number of computation instructions and messages to exchange, better will be your speed-up !
        \item \alert{Granularity can be improved with non blocking data exchanges}.
    \end{itemize}
\end{frame}

\begin{frame}[fragile]{Load balancing}
    \scriptsize
    \begin{block}{Definition}
        \textbf{All processes execute a computation section of the code in the same time};
    \end{block}

    \begin{itemize}
        \item Great penalities in time occure if some parts of the code are far away from load balancing;
        \item \textbf{Example 1} : A function in the code take $t$ seconds for the half of the processes, and $\frac{t}{2}$
              for other processes. The maximal speed-up for this function will be :
            \[
                S(n) = \frac{\frac{n}{2}t + \frac{n}{2}\frac{t}{2}}{t} = \frac{3n}{4}
            \]
        \item \textbf{Example 2} : A function in the code takes $\frac{t}{2}$ seconds for $n-1$ processes, and $t$
            for one process. The maximal speed-up for this function will be :
          \[
              S(n) = \frac{(n-1)\frac{t}{2} + t}{t} = \frac{n-1}{2} + 1 = \frac{n+1}{2}
          \]
  \end{itemize}

  \textbf{Remark} : Greater is the time token to execute a  bad load balancing function, greater is the penality. Don't worry
  about load balancing for functions taking very small time to execute
\end{frame}

\section{Embarrassingly parallel algorithms}
\subsection{Definition and properties}
\begin{frame}[fragile]{Definition}
    \scriptsize
    \begin{block}{Embarrassingly parallel algorithm}
        \begin{itemize}
            \item Each data used and computed are independant;
            \item No data race in multithread context;
            \item No communication between processes in distributed environment
        \end{itemize}
    \end{block}

    \begin{exampleblock}{Property}
        \begin{itemize}
            \item In distributed parallel context, no data must be exchanged between processes to compute the results;
            \item In shared parallel environment, parallelization is straightforward, but beware to the memory bound computation;
            \item In distributed environment, the memory bound limitation is not a trouble;
            \item If data are contiguous and algorithm vectorizable, can be ideal on GPGPU for performance.
        \end{itemize}
    \end{exampleblock}
\end{frame}

\begin{frame}[fragile]{First example : Vector addition}
    \scriptsize
    \begin{center}\small Add two real vectors of dimensions $N$ \end{center}

    \[
        w = u + v;\; u,v,w\in\mathbb{R}^{3}
    \]

    \begin{block}{Ideas}
    \begin{itemize}
        \item For load balancing, scatter the vectors in equal parts among the threads or processes 
        \item Each process/thread computes a part of the vector addition
        \item \alert{In distributed memory, the result is scattered among processes !}
    \end{itemize}
    \end{block}

    \begin{block}{Some properties}
    \begin{itemize}
        \item Memory access and computing operation have the same complexity : 
              \textbf{On shared memory, memory bound limits the performance}
        \item On distributed memory, each process uses his own physical memory and none data must be 
              exchanged : Speed-up may be linear relative to the number of processes (if data
              intensity is enough)
    \end{itemize}
    \end{block}
\end{frame}

\begin{frame}[fragile]{Example : Block diagonal matrices multiplication $C = A.B$ (1)}
    \scriptsize
    Matrix-matrix multiplication $C = A.B$ where
        \[
            A = \left(
                \begin{array}{cccc}
                    A_{11} & 0 & \ldots & 0 \\
                      0    & A_{22} & \ddots & 0 \\
                      \vdots & \ddots & \ddots & \vdots \\
                      0 & \ldots & 0 & A_{nn} 
                \end{array}\right),
            B = \left(
                \begin{array}{cccc}
                    B_{11} & 0 & \ldots & 0 \\
                      0    & B_{22} & \ddots & 0 \\
                      \vdots & \ddots & \ddots & \vdots \\
                      0 & \ldots & 0 & B_{nn} 
                \end{array}\right).
        \]
    where $d_{i} = \mbox{dim}(A_{ii}) = \mbox{dim}(B_{ii})$ ($n$ independant matrix-matrix multiplications)

    \begin{block}{\small Problematic}
        Close to the vector addition multiplication, but :
        \begin{itemize}
            \item Dimensions of diagonal blocks are inhomogeneous \& for each diagonal 
                  block, computation complexity : $d_{i}^{3}$.
            \item How distribute diagonal blocks among processes to obtain nearly optimal load balancing ?
        \end{itemize}
    \end{block}
\end{frame}

\begin{frame}[fragile]{Example : Block diagonal matrices multiplication $C = A.B$ (2)}
    \scriptsize
    \begin{center}\small algorithm to distribute diagonal blocks among processes\end{center}

    \begin{block}{Example of algorithm to distribute the diagonal blocks}
    \begin{itemize}
        \item Sort diagonal blocks with dimension decreasing;
        \item Set "weight" to zero for each process
        \item Distribute biggest triplet blocks $A_{ii},B_{ii},C_{ii}$ among processes and add each $d_{i}$ in the weight of each process;
        \item While some diagonal blocks are not distributed :
        \begin{itemize}
            \item {\scriptsize Add the biggest block which is not distributed to the process having the smallest weight }
            \item {\scriptsize Add the relative $d_{i}$ at the weight of the process}
        \end{itemize}
    \end{itemize}
    \end{block}

    \textbf{Remark} : All processes compute the distribution of the diagonal blocks. It's better to do same computation for all processes
                      than one computes and sends result to other processes.

\end{frame}

\begin{frame}[fragile]{Third example : Syracuse series (1)}
\scriptsize 
\begin{block}{\small Definition of syracuse serie}
    \[
        \left\{
        \begin{array}{l}
            u_{0} \mbox{ choosen} \\
            u_{n+1} = \left\{
                \begin{array}{l}
                    \frac{u_{n}}{2} \mbox{ if } u_{n}\mbox{ is even}\\[1mm]
                    3.u_{n}+1 \mbox{ if } u_{n} \mbox{ is odd.}
                \end{array}
            \right.
        \end{array}
        \right.
    \]
\end{block}

\begin{exampleblock}{\small Property of syracuse serie}
    \begin{itemize}
        \item One cycle exists : $1 \rightarrow 4 \rightarrow 2 \rightarrow 1 \rightarrow \cdots$
        \item A conjecture : \alert{$\forall u_{0}\in \mathbb{N}$, 
              the serie reaches the cycle above in a finite number of iterations}
    \end{itemize}
\end{exampleblock}

\begin{alertblock}{\small Some definition}
    \begin{itemize}
        \item \textcolor{blue}{\bf Length of flight} : number of iterations for a serie to reach the value 1;
        \item \textcolor{NavyBlue}{\bf Height of the flight} : maximal value reached by a serie
    \end{itemize}
\end{alertblock}
\begin{center}{\bf The goal of the program : } compute the length and the height of flight for lot of 
    odd values for $u_{0}$
\end{center}
\end{frame}

\begin{frame}[fragile]{Third example : Syracuse series (2)}
    \scriptsize
    \begin{block}{\small Problematic}
        \begin{itemize}
            \item Each process computes the length and the height for a subset of initial values $u_{0}$;
            \item The computation intensity depends of the length of each syracuse serie;
            \item It's impossible to know a priori the computation complexity to compute the height and the length of a serie
            \item Use a dynamic algorithm on "root" process (the "master" process) to distribute serie among other processes ("slaves")
        \end{itemize}
    \end{block}

    \begin{multicols}{2}
        \begin{exampleblock}{\small Master's Algorithm}
            \begin{itemize}
                \item Send a small pack of series to each slave processes;
                \item While(some pack of series to send) do 
                \item \hspace*{2mm}Wait slave asking series and send next pack;
                \item end While
                \item Send terminaison id to all slave processes;
            \end{itemize}
        \end{exampleblock}
    

        \begin{exampleblock}{\small Slave's Algorithm}
            \begin{itemize}
                \item While (receive some series to compute in a pack)
                \item \hspace*{2mm}Compute each serie of the pack;
                \item end While
            \end{itemize}
        \end{exampleblock}
    \end{multicols}

    \textbf{Remark} : A task is composed of a pack of series to have a good granularity.

\end{frame}

\section{Nearly embarrasingly parallel algorithm}

\begin{frame}[fragile]{Nearly embarrasingly parallel algorithm}
    \scriptsize
    \begin{block}{\small Definition}
        Independant computation for each processes with a final communication to finalize the computation.
    \end{block}

    \begin{exampleblock}{\small Examples}
        \begin{itemize}
            \item Dot product of two vector in $\mathbb{R}^{n}$;
            \item Compute an integral;
            \item Matrix-vector product;
        \end{itemize}
    \end{exampleblock}

    \begin{alertblock}{\small Non embarrasingly parallel algorithm examples}
      \begin{itemize}
      \item Parallel sort algorithms;
      \item Matrix-matrix product;
      \item Algorithms based on domain decomposition methods;
      \end{itemize}
    \end{alertblock}
\end{frame}

\begin{frame}[fragile]{Integral computation}
  \scriptsize
  \begin{block}{\small Integral computation}
    \begin{itemize}
    \item Integral computation based on Gauss quadrature formulae :
      \[
      \int_{a}^{b}f(x)dx \approx \sum_{i=1}^{Ng} \omega_{i}f(g_{i})
      \]
      where $\omega_{i}\in\mathbb{R}$ the weights  and $g_{i}\in\mathbb{R}$ integrals points.
    \item In fact, Gauss quadrature are given on $[-1;1]$ interval : some variable modification to do in the integral !;
    \item $\left\{ g_{1}=0, \omega_{1}=2 \right\}$ : Order 1 Legendre Gauss quadrature;
    \item $\left\{ \left( g_{1}=-\frac{\sqrt{2}}{2}, \omega_{1}=\frac{5}{9} \right), \left( g_{2}=0, \omega_{2}=\frac{8}{9} \right), \left( g_{3}=+\frac{\sqrt{2}}{2}, \omega_{3}=\frac{5}{9} \right) \right\}$ : Order 3 Legendre Gauss quadrature
    \item \textbf{Remark} : Order $n$ means than quadrature compute the exact value of the integral for polynoms of degree lesser or equal to $n$.
    \item To compute better approximation of the integral, we subdivide the interval in several smaller intervals
    \end{itemize}
  \end{block}
\end{frame}

\begin{frame}[fragile]{Parallel integral computation}
  \scriptsize
  \begin{center}
    \[
    I = \int_{a}^{b}f(x)dx = \sum_{i=1}^{N}\int_{a_{i}}^{b_{i}}f(x)dx = \sum_{i=1}^{N} I_{î} \mbox{ where } a_{1}=a, b_{N} = b\mbox{ and } a_{i} < b_{i} < a_{i+1}
    \]
  \end{center}
  \begin{block}{\small Main ideas}
    \begin{itemize}
  \item Scatter sub-intervals among the processes $P$ to compute partial sums :
    \[
    S_{p} = \sum_{[a_{i};b_{i}] \in P} \int_{a_{i}}^{b_{i}} f(x)dx
    \]
  \item Use reduce to compute the integral value (global sum) :
    \[ S = \sum_{p=1}^{nbp} S_{p} \]
    \end{itemize}
  \end{block}
\end{frame}

\begin{frame}[fragile]{Matrix-vector product}
  \scriptsize
  Let's a matrix $A\in\mathbb{R}^{m\times n}$ and a vector $u\in\mathbb{R}^{m}$.

  The goal of this algorithm is to compute the matrix-vector product :
  \[
  v = A.u\in\mathbb{R}^{n}\mbox{ where } v_{i} = \sum_{j=1}^{m} A_{ij}.u_{j}
  \]
  
  Two possibilities to parallize this algorithm :
  \begin{itemize}
  \item Partitionning the matrix per block of rows :
  \item Partitionning the matrix per block of columns and the vector $u$ per block of same size.
  \end{itemize}

  The goal is to split the computation between processes and use a global communication operation
  to get the final result.

\end{frame}

\begin{frame}[fragile]{Matrix-vector product per rows splitting}
  \scriptsize
  Let's
  \[
  A = \left(\begin{array}{c}
    A_{1} \\ \hline
    A_{2} \\ \hline
    \vdots \\ \hline
    A_{I} \\ \hline
    \vdots \\ \hline
    A_{N}
    \end{array}
  \right)
  \mbox{ where } \forall I\in\left\{1,2,\ldots,N\right\}, A_{I}\in\mathbb{R}^{\frac{n}{N}\times m}.
    \]

    \begin{exampleblock}{Algorithm}
      \begin{itemize}
      \item Each process computes a part of $v$. The process $I$ computes 
        $\displaystyle
        V_{I} = A_{I}.u \in \mathbb{R}^{\frac{n}{N}}
        $ 
      \item To compute another matrix-vector product with the new vector, we need to gather the vector
            in all processes (only necessary for distributed parallel algorithm).
      \end{itemize}
    \end{exampleblock}
  
\end{frame}

\begin{frame}[fragile]{Matrix-vector product per columns splitting}
  \scriptsize
  Let's
  \[
    A = \left(A_{1}|A_{2}|\ldots|A_{I}|\ldots|A_{N}\right)\mbox{ and }
    u=\left(\begin{array}{c}
      U_{1} \\ \hline
      U_{2} \\ \hline
      \vdots \\ \hline
      U_{I} \\ \hline
      \vdots \\ \hline
      U_{N}
    \end{array} \right)
    \mbox{ where } \forall I\in\left\{1,2,\ldots,N\right\}, A_{I}\in\mathbb{R}^{n\times \frac{m}{N}}\mbox{ and } U_{I}\in\mathbb{R}^{\frac{m}{N}}
    \]

    \begin{exampleblock}{Algorithm}
      \begin{itemize}
      \item Each process computes a partial sum for $v$. Processus $I$ computes
        \[
        V_{I} = A_{I}.U_{I}\in\mathbb{R}^{n}
        \]
      \item Finnaly, a sum reduction is done to get the final result :
        $\displaystyle
        v = \sum_{I=1}^{N}V_{I}
        $
      \end{itemize}
    \end{exampleblock}
    
\end{frame}

\begin{frame}[fragile]{Bhudda set}
  \scriptsize
  \begin{multicols}{2}
  Let's consider the complex recursive mandelbrot serie :
  \[
  \left\{
  \begin{array}{l}
  z_{0}=0,\\
  z_{n+1} = z_{n}^{2} +c \mbox{ where } c\in\mathbb{C} \mbox{ chosen.}
  \end{array}\right.
  \]

  \begin{figure}[h]
    \includegraphics[width=0.3\linewidth]{../Images/mandelbrot.jpeg}
    \includegraphics[width=0.2\linewidth]{../Images/bhudda.jpg}
    \caption{\small Mandelbrot (left) and Bhudda's (right) set}
  \end{figure}
  \end{multicols}

  \begin{multicols}{2}
  \begin{block}{\small Property}
  \begin{itemize}
  \item Serie divergent if $\exists n>0; |z_{n}|>2$;
  \item Region of interest : the disk $\mathcal{D}$ of radius 2;
  \item In some region of the disk, possible to prove the convergence;
  \item But chaotic convergence behaviour in some region of $\mathcal{D}$ !
  \end{itemize}
  \end{block}

  \begin{exampleblock}{\small Mandelbrot and bhudda sets}
    \begin{itemize}
    \item \textcolor{blue}{Mandelbrot's set} : \\ color $c$ with  "divergence speed" of relative series
    \item \textcolor{DarkGreen}{Bhudda's set} : \\ Color  orbit of divergent series
    \end{itemize}
  \end{exampleblock}
  \end{multicols}
\end{frame}

\begin{frame}[fragile]{Bhudda's set algorithm}
    \begin{exampleblock}{Algorithm}
        \begin{itemize}
            \item Draw $N$ random values of $c$ in the disk $\mathcal{D}$ where the relative series diverge;
            \item Compute the orbit of this serie until divergence and increment the intensity of the pixel
                  representing each value of the orbit;            
        \end{itemize}
    \end{exampleblock}

    \begin{alertblock}{Parallelization of the algorithm}
        \begin{itemize}
            \item Master-slave algorithm to ensure load balancing;
            \item For granularity, one define a task as a pack of random values $c$;
        \end{itemize}
    \end{alertblock}
\end{frame}


\end{document}
